\chapter[Referências Bibliográficas]{Referências Bibliográficas}

Agile Manifesto (2001) Manifesto para Desenvolvimento Ágil de Software. Disponível em: <http://agilemanifesto.org/iso/ptbr/>. Acesso em: 02 nov. 2016.

AMBLER, S. W. Agile Requirements Best Practices. Agile Modeling. Disponível em: <http://www.agilermodeling.com/essays/agileRequirementsBestPractices.htm>. Acesso em 02 nov. 2016.

AMBLER, S. W. Introduction to User Stories. Agile Modeling. Disponível em: <http://www.agilermodeling.com/artifacts/userStory.htm>. Acesso em 02 nov. 2016.

ASSOCIAÇÃO BRASILEIRA DE NORMAS TÉCNICAS. NBRISO/IEC9126-1 Engenharia de software - Qualidade de produto - Parte 1: Modelo de qualidade. 2003.

AURUM, A., WOHLIN, C., Engineering and Managing Software Requirements, Springer-Verlag, 2005. 

BELGAMO, Anderson; Martins, Luiz. E. G; Um Estudo Comparativo sobre as Técnicas de Elicitação de Requisitos do Software, XIX CTIC (Concurso de Trabalhos de Iniciação Científica) no XX Congresso Brasileiro da Sociedade Brasileira de Computação, Curitiba, 2000.

GOTEL, O. e FINKELSTEIN, A., “An analysis of the Requirements Traceability Problem,” in Proceedings of the First International Conference on Requirements Engineering, (Colorado springs, CO), pp. 94-101, April 1994.

KIRAWOWSKI, J; Requirements Engineering and Specification in Telematics: Methods for User-Orientated Requirements Specification. Human Factors Research Group, Cork, Ireland,1997. Disponível em http://www.ejeisa.com/ nectar/respect/3.2/.

KOTONYA, G., SOMMERVILLE, I., Requirements engineering: processes and
techniques. Chichester, England: John Wiley, 1998. 

LEFFINGWELL, D. Agile Software Requirements: Lean Requirements Practices for Teams, Programs, and the Enterprise, Capítulo 12, 2010.

MENDES, L. M.; COSTA, R. H.; LOURENSO, R. O Gerenciamento de Requisitos e a sua Importância em Projetos de Desenvolvimento de Software. Instituto Federal de Educação, Ciência e Tecnologia de São Paulo, 2015.

POHL, Klaus; Rupp, Chris. “Requirements engineering fundamentals: A study guide for the certified professional for Requirements Engineering exam: Foundation level, IREB compliant”. 1.ed, 2012

Portal DevMedia. Disponível em: http://www.devmedia.com.br/engenharia-de-software-2-tecnicas-para-levantamento-de-requisitos/9151. Acesso em 25 de outubro de 2016.

ROBERTSON, S., ROBERTSON, J. Mastering the Requirements Process. 2nd Edition. Addison Wesley, 2006.

SAFe Framework. Disponível em: http://www.scaledagileframework.com/. Acesso em 25 de outubro de 2016.

SILLITTI, A.; HAZZAN, O.; BACHE, E.; ALBALADEJO, X. Agile Processes in Software Engineering and Extreme Programming: 12th International Conference, XP 2011, Madrid, Spain, May 10-13, 2011.

SOMMERVILLE, I. Engenharia de Software, 8a edição. Tradução: Selma Shin Shimizu Melnikoff, Reginaldo Arakaki, Edilson de Andrade Barbosa; São Paulo: Pearson Addison-Wesley, 2007.

WIEGERS, K.E., Software Requirements: Practical techniques for gathering and managing requirements throughout the product development cycle. 2nd Edition, Microsoft Press, Redmond, Washington, 2003. 
