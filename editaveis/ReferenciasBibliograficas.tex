\chapter[Referências Bibliográficas]{Referências Bibliográficas}

BELGAMO, Anderson; Martins, Luiz. E. G; Um Estudo Comparativo sobre as Técnicas de Elicitação de Requisitos do Software, XIX CTIC (Concurso de Trabalhos de Iniciação Científica) no XX Congresso Brasileiro da Sociedade Brasileira de Computação, Curitiba, 2000.

GOTEL, O. e FINKELSTEIN, A., “An analysis of the Requirements Traceability Problem,” in Proceedings of the First International Conference on Requirements Engineering, (Colorado springs, CO), pp. 94-101, April 1994.

KIRAWOWSKI, J; Requirements Engineering and Specification in Telematics: Methods for User-Orientated Requirements Specification. Human Factors Research Group, Cork, Ireland,1997. Disponível em http://www.ejeisa.com/ nectar/respect/3.2/. 

LEFFINGWELL, D. Agile Software Requirements: Lean Requirements Practices for Teams, Programs, and the Enterprise, Capítulo 12, 2010.

MENDES, L. M.; COSTA, R. H.; LOURENSO, R. O Gerenciamento de Requisitos e a sua Importância em Projetos de Desenvolvimento de Software. Instituto Federal de Educação, Ciência e Tecnologia de São Paulo, 2015.			

POHL, Klaus; Rupp, Chris. “Requirements engineering fundamentals: A study guide for the certified professional for Requirements Engineering exam: Foundation level, IREB compliant”. 1.ed, 2012 

SILLITTI, A.; HAZZAN, O.; BACHE, E.; ALBALADEJO, X. Agile Processes in Software Engineering and Extreme Programming: 12th International Conference, XP 2011, Madrid, Spain, May 10-13, 2011.

Portal DevMedia. Disponível em: http://www.devmedia.com.br/engenharia-de-software-2-tecnicas-para-levantamento-de-requisitos/9151. Acesso em 25 de outubro de 2016.

SAFe Framework. Disponível em: http://www.scaledagileframework.com/. Acesso em 25 de outubro de 2016.

