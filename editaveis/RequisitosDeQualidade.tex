\section[Requisitos de Qualidade]{Requisitos de Qualidade}
A ISO/IEC 9126 é uma família de normas que trata da avaliação da qualidade de
produtos de software. Ela estabelece um modelo de qualidade para produtos de software e
apresenta diversas medidas para aferir qualitativa e quantitativamente a presença dos atributos
de qualidade descritos em seu modelo. 

O modelo de qualidade interna e externa especifica seis características, sendo elas:
\begin{quote} 
	\begin{itemize}
        \item Funcionalidade
		\item Confiabilidade
		\item Usabilidade
		\item Eficiência
		\item Manutenibilidade
		\item Portabilidade
    \end{itemize}
\end{quote}

Baseando-se na ISO, a equipe procurou atender algumas características da seguinte forma:
\begin{quote} 
	\begin{itemize}
        \item Funcionalidade: adequação das funções do produto com o usuário que irá utilizá-lo;
        \item Usabilidade: a partir da apreensibilidade e atratividade, para que o usuário consiga aprender como operar e controlar o produto;
        \item Eficiência: baseando-se no comportamento em relação ao tempo de resposta e o tempo de processamento;
        \item Eficácia: atingir as metas especificadas;
        \item Produtividade: permitindo ao usuário usar os recursos apropriados em relação à eficácia;
        \item Segurança: dados e informações;
        \item Satisfação: dos usuários no contexto de uso específico.
    \end{itemize}
\end{quote}