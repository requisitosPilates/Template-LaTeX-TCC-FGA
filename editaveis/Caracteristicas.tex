\section[Caracteristicas]{Caracteristicas}
O desenvolvimento de projetos de software ainda passa por processos de adaptação. A metodologia ágil aparece nesses ambientes mais complexos, focada na análise crítica de valor agregado e vantagem competitiva do cliente. O principal objetivo é o fato de entregar rapidamente as necessidades de maior valor agregado, ou seja, focar no que é essencial no desenvolvimento do projeto, evitando o excesso de documentação.

A parte de elicitação de requisitos deve refletir a necessidade de um cliente e ao mesmo tempo deve ser clara para os desenvolvedores. Os requisitos expressam as funcionalidades do sistema, assim esclarece Sommerville (2007): as definições de requisitos de sistema especificam o que o sistema deve fazer (suas funções) e suas propriedades essenciais e desejáveis. […] Esses requisitos refletem as necessidades dos clientes de um sistema que ajuda a resolver algum problema.

Segundo Moraes (2009, p.55), o levantamento de requisitos inadequados pode impossibilitar o rastreamento das causas de problemas, custos maiores que o planejado, prazos acima do estimado e processos fundamentais ao cliente omissos.
De acordo com o fator crítico, os requisitos estão diretamente relacionados à qualidade do software: “A qualidade de um software pode ser definida como a conformidade aos requisitos.” (KOSCIANSKI, 1992).

Diante destes desafios, o Manifesto Ágil descreve como os requisitos devem ser especificados de acordo com os seguintes valores:
\begin{quote} 
    \begin{itemize}
        \item Indivíduos e interações: mais que processos e ferramentas
        \item Software em funcionamento: mais que documentação abrangente
        \item Colaboração com o cliente: mais que negociação de contratos
        \item Responder a mudanças: mais que seguir um plano
    \end{itemize}
\end{quote}

Como os requisitos são descritos em metodologias ágeis?
AMBLER (2004-2007) descreve os seguintes conselhos aos Analistas de Requisitos em projetos de desenvolvimento de software em um ambiente ágil:
\begin{quote} 
    \begin{itemize}
        \item No início do projeto, compreenda o escopo e gere requisitos de alto nível. Nesse momento os detalhes não são importantes. Esse levantamento inicial não deve chegar a durar semanas, apenas dias.
        \item Idealmente, os detalhes são modelados e/ou analisados durante o tempo (just-in-time), através de sessões envolvendo poucas pessoas para esclarecer as questões necessárias para o desenvolvimento.
        \item Reconheça que a análise dos requisitos é realizada durante todo o projeto e não há mais a “fase de análise”.
        \item O requisito pode mudar a qualquer momento, o que é normal e aceitável. É necessário gerenciá-los conforme suas prioridades.
        \item Adote modelagens para que os stakeholders estejam aptos a entender, incluindo as modelagens mais técnicas.
        \item Analistas de requisitos efetivos sabem aplicar várias técnicas de modelagem em sistemas complexos.
        \item O objetivo é entender os requisitos e não gerar documentos de requisitos, mas caso o documento seja escrito mantenha-o atualizado e útil para os envolvidos, tendo uma única fonte de informação.
        \item Reconheça que há vários modelos a sua disposição.
        \item Modele com os desenvolvedores, para que eles entendam os requisitos e você entenda o que eles estão tentando construir.
        \item Os melhores analistas de requisitos são os especialistas generalizados que possuem mais de uma especialidade além da de levantar requisitos.
    \end{itemize}
\end{quote}

Baseado em SOMMERVILLE (2007), SMITH (2009) e AMBLER (2004-2007), no desenvolvimento iterativo não há especificação detalhada e o documento descreve as características mais importantes do sistema. Em processo tradicional de desenvolvimento, deve-se identificar todos os requisitos. Já em processo ágil, apenas o suficiente para identificar o que será construído. Desta forma, o grupo optou pelo início da modelagem de requisitos Épicos, Features e Histórias de usuário.

História de usuário é uma substituição ágil dos documentos de requisitos, como o caso de uso, por exemplo, elas possuem breves descrições de algo que o sistema precisa fazer para o usuário.  (LEFFINGWELL 2011, p. 57).

As histórias de usuário são ferramentas para definir o comportamento de um sistema de um jeito que seja compreensível para o desenvolvedor e os usuários e suas descrições focam no valor definido por ele, fornecendo uma abordagem leve e eficaz para o gerenciamento dos requisitos. (LEFFINGWELL 2011, p. 100).

Além disso, foram adotadas boas práticas para auxiliar a modelagem e documentação dos requisitos, como diz AMBLER (2004-2007):
\begin{quote} 
    \begin{itemize}
        \item Os stakeholders participaram ativamente;
        \item Adoção de uma abordagem ampla para ter uma visão geral do sistema;
        \item Modelagem os detalhes just-in-time (modelagem evolutiva);
        \item Tratamento de prioridades: para atender à mudanças e criação de novos requisitos;
        \item Preferência de implementar os requisitos e documentar apenas o suficiente;
        \item Investimento na matriz de rastreabilidade;
        \item Adoção das terminologias do stakeholder;
        \item Conexão entre o stakeholder e os desenvolvedores.
    \end{itemize}
\end{quote}