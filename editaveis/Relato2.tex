\section[Relato de experiência da disciplina de Engenharia de Requisitos]{Relato de experiência da disciplina de Engenharia de Requisitos}
A Engenharia de Requisitos (ER) é o processo pelo qual os requisitos de um produto de software são coletados, analisados, documentados e gerenciados ao longo de todo o ciclo de vida do software (AURUM; WOHLIN, 2005).

O processo de engenharia de requisitos envolve criatividade, interação de diferentes pessoas, conhecimento e experiência para transformar informações diversas (sobre a organização, sobre leis, sobre o sistema a ser construído etc.) em documentos e modelos que direcionem o desenvolvimento de software (KOTONYA; SOMMERVILLE, 1998). 

A disciplina de ER é uma das mais importantes dentro do curso de Engenharia de Software. E a forma como ele é administrada impacta bastante
na formação dos alunos. A partir dela, o aluno adquire conhecimento de diversas atividades que devem ser realizadas:

\begin{quote} 
	\begin{itemize}
        \item Atividades de Desenvolvimento (ou Técnicas) que contribuem diretamente no processo de desenvolvimento do software, como elicitar, analisar e implementar os requisitos;
        \item Atividades de Gerência que estão relacionadas às atividades de planejamento e acompanhamento do projeto, como realização de estiamtivas, elaboração de cronogramas e análise de riscos;
        \item Atividades de Controle da Qualidade que são responsáveis pela avaliação do produto. Envolvem as atividades de verificação, validação e garantia da qualidade.
    \end{itemize}
\end{quote}

A ER não trata apenas do elicitação e desenvolvimento de
requisitos, ela também lida com toda a parte de gerência.

Saber elicitar bem os requisitos influencia diretamente na qualidade do software a ser desenvolvido. A disciplina ao propor uma mistura de eoria e prática, direciona o aluno a melhor maneira de se fazer isso. Pelo fato de trabalhar com um problema real e um cliente real, a parte prática da disciplina se torna mais interessante e válida.
