\chapter[Técnicas de Elicitação de Requisitos]{Técnicas de Elicitação de Requisitos}
A primeira fase para o desenvolvimento de um software é o levantamento de
requisitos, ou seja, é quando se obtém informações do cliente sobre o que ele
deseja que seja construído. É a atividade responsável para definir as necessidades
de sistema e as características que o mesmo deve ter. São levantadas questões como:
funcionalidades que o sistema deve ter, regras de negócio, limitações, usabilidade
de software, organização da empresa e assim por diante.

Sommerville (2003) propõe um processo genérico de levantamento e análise com as
seguintes atividades: Compreensão de domínio; Coleta de requisitos;
Classificação; Resolução de conflitos; Definição de prioridades; e, por fim,
Verificação de requisitos.

Para evitar que problemas nesta fase de um projeto, são utilizadas as
técnicas de elicitação de requisitos. Quando não é utilizada a técnica adequada
 e a descrição de sistema não é conciso e consistente, o projeto enfrenta
dificuldades. Então, o objetivo das técnicas é superar essas possíveis
dificuldades encontradas.
A partir do Relatório 1, foram escolhidas algumas técnicas para o
desenvolvimento de sistema: brainstorming, entrevista e prototipação.

\section[Brainstorming]{Brainstorming}
O \textsl{brainstorming} foi feito durante as reuniões iniciais, tanto com
time quanto com a cliente. Onde o problema era levantado e todos poderiam propor
uma solução. Após isso era feita um discussão para selecionar a melhor proposta.

É interessante comentar que o \textsl{brainstorming} mantém uma dinâmica muito amigável e sem formalidades, dismistificando e quebrando qualquer inconveniente entre os stakeholders.

\section[Entrevista]{Entrevista}
Essa técnica foi feita durante todas as reuniões com a cliente a fim de
elicitar as necessidades e prioridades. Ficou claro durante o uso dessa
técnica que o foco deveria ser na parte financeira. Quando surgia alguma dúvida, o grupo entrava em contato com a cliente e a mesma foi solícita.

A entrevista se mostrou muito útil e aplicável na maioria dos casos de elicitação de requisitos, justamente por ser simples e ágil, além de aproximar o cliente e os analistas de requisitos.

\section[Prototipação]{Prototipação}
A prototipação, item planejado no  Relatório 1 da disciplina de ER, acabou por se tornar não necessária por muitos fatores, são eles:
\begin{quote} 
	\begin{itemize}
        \item A cliente esteve sempre próxima ao desenvolvimento, facilitando a resolução de problemas;
		\item A cliente já teve uma experiência ruim com outro software e relatou para a equipe os problemas enfrentados para evitar que os mesmos existam na nova solução apresentada;
		\item A prototipação planejada, apesar de ser de baixa fidelidade, demandaria tempo desnecessário em contrapartida da alta rapidez em que a ferramenta de desenvolvimento escolhida proporciona.
    \end{itemize}
\end{quote}