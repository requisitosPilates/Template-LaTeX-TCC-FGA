\chapter[Técnicas de Elicitação de Requisitos]{Técnicas de Elicitação de Requisitos}
A primeira fase para o desenvolvimento de um software é o levantamento de requisitos, ou seja, é quando se obtém informações do cliente sobre o que ele deseja que seja construído. É a atividade responsável para definir as necessidades de sistema e as características que o mesmo deve ter. São levantadas questões como: funcionalidades que o sistema deve ter, regras de negócio, limitações, usabilidade de software, organização da empresa e assim por diante.

Sommerville (2003) propõe um processo genérico de levantamento e análise com as seguintes atividades: Compreensão de domínio; Coleta de requisitos; Classificação; Resolução de conflitos; Definição de prioridades; e, por fim, Verificação de requisitos.

Para evitar que problemas nesta fase de um projeto, são utilizadas as técnicas de elicitação de requisitos. Quando não é utilizada a técnica adequada e a descrição de sistema não é conciso e consistente, o projeto enfrenta dificuldades. Então, o objetivo das técnicas é superar essas possíveis dificuldades encontradas.
A partir do Relatório 1, foram escolhidas algumas técnicas para o desenvolvimento de sistema: brainstorming, entrevista e prototipação.
