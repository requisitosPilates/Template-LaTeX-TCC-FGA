\section[UserStories]{UserStories}
\subsection[Visualizar Agenda Diária]{Visualizar Agenda Diária}
Eu como usuário administrador quero visualizar agenda diária para
me certificar de quais alunos tiveram ou terão acesso ao estabelecimento
em determinada data para aula.

\subsection[Visualizar Horário Específico]{VisualizarHorário Específico}
Eu como usuário administrador quero visualizar a agenda referente a um horário
específico para me certificar de quais alunos tiveram ou terão acesso ao
estabelecimento em determinado momento do dia para aula.

\subsection[Visualizar Agenda Semanal]{Visualizar Agenda Semanal}
Eu como usuário administrador quero visualizar a agenda semanal para assim me
certificar de quais alunos tiveram ou terão acesso ao estabelecimento durante tal
semana para aulas.

\subsection[Visualizar Agenda Mensal]{Visualizar Agenda Mensal}
Eu como usuário administrador quero visualizar a agenda mensal para me
certificar de quais alunos tiveram ou terão acesso ao estabelecimento durante
determinado mês para aulas.

\subsection[Gerar Relatório Diário de Aulas]{Gerar Relatório Diário de Aulas}
Eu como usuário administrador quero gerar um relatório diário de aulas para me
certificar de quais alunos compareceram ou não a aula agendada.

\subsection[Criar Aulas]{Criar Aulas}
Essa história de usuário tem como objetivo o desenvolvimento da \textsl{model} referente
a Aulas. Assim permitindo que sejam feitas dependências internas entre aluno
e aulas.

\begin{quote}
Critérios de aceitação:
    \begin{itemize}
        \item O sistema deve permitir registrar novas aulas.
    \end{itemize}
\end{quote}

\subsection[Cadastrar Plano de Contrato]{Cadastrar Plano de Contrato}
Eu como usuário administrador quero cadastrar planos de contratos com as
informações de: quantidade de vezes por semana, período do plano em meses,
valor mensal e percentual de desconto baseado no plano de 1 mês.

\begin{quote}
Critérios de Aceitação:
    \begin{itemize}
        \item O sistema deve permitir a edição de um ou mais dados de planos de contrato
        como o preço da mensalidade e o desconto.
    \end{itemize}
\end{quote}

\subsection[Submeter Contrato]{Submeter Contrato}
Eu como usuário administrador quero realizar o \textsl{upload} de contratos nos formatos
PDF, doc com seu respectivo título e descrição para que eu possua um controle
sobre os antigos contratos e possa imprimir os atuais.

\subsection[Visualizar Contrato]{Visualizar Contrato}
Eu como usuário administrador quero visualizar a lista dos documentos hospedados
(listados pelo seu respectivo título) e abrir esses documentos para que eu possa
visualizar uma lista dos arquivos disponíveis no sistema e realizar o \textsl{download}
de cada um deles.

\subsection[Remover Contrato]{Remover Contrato}
Eu como usuário administrador quero remover do sistema os contratos hospedados
para que eu possa remover os que não são mais importantes e possua controle
sobre os contratos.

\subsection[Cadastrar Professor]{Cadastrar Professor}
Eu como usuário administrador quero cadastrar os professores no sistema para
que possa ter um registro com as informações dos professores e maior controle
sobre eles.

\subsection[Alterar Dados de Professor]{Alterar Dados de Professor}
Eu como usuário administrador quero alterar dados dos professores no sistema
para que possa atualizar o registro que contém as informações sobre eles.

\subsection[Alterar Status de Professor]{Alterar Status de Professor}
Eu como usuário administrador quero registrar no sistema se o professor está
ativo ou não no estúdio de pilates, a fim de controlar a alocação de
professores em horários de aulas e permitir ou rejeitar o seu acesso ao sistema.

\subsection[Cadastrar Saídas]{Cadastrar Saídas}
Eu como usuário administrador quero cadastrar as saídas(despesas) da empresa para que
possa manter um histórico financeiro da empresa.
Exemplo: Contas de água, luz, telefone, internet, impostos.

\subsection[Editar Saídas]{Editar Saídas}
Eu como usuário administrador quero editar as saídas(despesas) da empresa para que possa
corrigir qualquer eventual equívoco ou falta de informações associadas ao
histórico financeiro da empresa.

\subsection[Visualizar Saída]{Visualizar Saída}
Eu como usuário administrador quero visualizar as saídas(despesas) da empresa
para que possa analisar o fluxo de caixa da empresa.

\subsection[Visualizar Receita]{Visualizar Receita}
Eu como usuário administrador quero visualizar as entradas e saídas(despesas) da empresa
para que possa analisar o fluxo de caixa da empresa.

\subsection[Gerar Relatório de Entradas]{Gerar Relatório de Entradas}
Eu como usuário administrador quero gerar um relatório que especifique as
entradas da empresa para que possa melhor gerencia-la e verificar o quanto de
dinheiro foi obtido pela empresa.

\subsection[Visualizar Entradas]{Visualizar Entradas}
Eu como usuário administrador quero visualizar as entradas para que possa
verificar o quanto de dinheiro foi obtido pela empresa.

\subsection[Agendar Aluno à Horário Fixo]{Agendar Aluno à Horário Fixo}
Eu como usuário administrador quero atribuir horários fixos de aula à alunos
cadastrados no sistema.

\subsection[Agendar Aluno à Aula de Reposição]{Agendar Aluno à Aula de Reposição}
Eu como usuário administrador quero atribuir novo horário a alunos que solicitam
reposição de aula.

\begin{quote}
Critérios de Aceitação:
    \begin{itemize}
        \item Indicar a data da aula que está sendo reposta.
    \end{itemize}
\end{quote}


\begin{quote}
Critérios de Aceitação:
    \begin{itemize}
        \item Ao selecionar a aula de algum aluno, deve ser possível criar uma nova aula,
        que será aula de reposição, transformando a aula atual em aula cancelada.
    \end{itemize}
\end{quote}

\subsection[Agendar Aluno à Aula]{Agendar Aluno à Aula}
Eu como usuário administrador quero atribuir um aluno a determinado
horário/aula quando necessário.

\subsection[Desmarcar Aula de Aluno]{Desmarcar Aula de Aluno}
Eu como usuário administrador quero cancelar aula de determinado aluno quando
necessário.

\subsection[Remarcar Aula de Aluno]{Remarcar Aula de Aluno}
Eu como usuário administrador quero remarcar aulas referentes a determinado
aluno quando necessário. Como por exemplo mudar aula de um dia para outro,
tanto pra dias anteriores como posteriores, mas já com uma data definida
para fazer a aula.

\subsection[Conferir Aulas Remanescentes]{Conferir Aulas Remanescentes}
Eu como usuário administrador quero conferir as aulas restantes de cada aluno.

\subsection[Conferir Aulas Realizadas]{Conferir Aulas Realizadas}
Eu como usuário administrador quero conferir as aulas realizadas por alunos.

\subsection[Agendar Aula Experimental]{Agendar Aula Experimental}
Eu como usuário administrador quero agendar aula experimental quando necessário.

\begin{quote}
Critérios de Aceitação:
    \begin{itemize}
        \item O sistema deve permitir criar uma aula experimental quando necessário direto
        no menu das aulas existentes.
        \item É necessário apenas o nome e o telefone do aluno que irá fazer a aula
        experimental.
    \end{itemize}
\end{quote}

\subsection[Atribuir Falta de Aluno à Aula Marcada]{Atribuir Falta de Aluno à Aula Marcada}
Eu como usuário administrador quero atribuir uma falta a um aluno quando não
houver aviso prévio.

\subsection[Conferir Aulas de Reposição de Aluno]{Conferir Aulas de Reposição de Aluno}
Eu como usuário administrador quero conferir quantas aulas referentes a
reposição (aulas desmarcadas) o aluno possui.

\subsection[Conferir Aulas Perdidas de Aluno]{Conferir Aulas Perdidas de Aluno}
Eu como usuário administrador quero conferir a quantidade de faltas que o aluno
possui.

\subsection[Colocar Aluno na Lista de Espera]{Colocar Aluno na Lista de Espera}
Eu como usuário administrador quero colocar um aluno na lista de espera quando o
mesmo estiver esperando por um horário específico enquanto não houver vaga.

\subsection[Alterar Horário de Aula de Aluno]{Alterar Horário de Aula de Aluno}
Eu como usuário administrador quero alterar o horário de aula de um aluno no
mesmo dia.

\subsection[Adicionar Plano de Contrato ao Aluno]{Adicionar Plano de Contrato ao Aluno}
*Eu como usuário administrador quero associar um aluno ao plano de contrato feito
com o mesmo: com as informações do plano (quantidade de vezes na semana,
mensalidade, período de meses e desconto) além do início, fim e forma de
pagamento (dinheiro ou cheque) explicitando o dia, mês, banco e número dos
cheques.

\subsection[Renovar Plano de Aluno]{Renovar Plano de Aluno}
*Eu como usuário administrador quero renovar plano de aluno ao fim do
contrato do mesmo.

\subsection[Visualizar Vencimento de Plano de Aluno]{Visualizar Vencimento de Plano de Aluno}
*Eu como usuário administrador quero visualizar quando o plano de determinado
aluno irá vencer.

\subsection[Gerar Relatório Mensal com o Vencimentos Planos]{Gerar Relatório Mensal com o Vencimentos Planos}
*Eu como usuário administrador quero gerar um relatório mensal onde seja possível
saber quais planos referentes a quais alunos devem vencer no mês.

\subsection[Enviar Notificações de Vencimento para o Aluno]{Enviar Notificações de Vencimento para o Aluno}
Eu como usuário administrador quero que sejam enviadas notificações aos alunos,
de forma automatizada, sobre o vencimento da sua mensalidade e/ou plano.

\subsection[Gerar Relatório Mensal com os Alunos Ativos]{Gerar Relatório Mensal com os Alunos Ativos}
Eu como usuário administrador quero gerar um relatório mensal onde eu posso me
certificar dos alunos ainda ativos no sistema.

\subsection[Cancelar Plano de Contrato de Aluno]{Cancelar Plano de Contrato de Aluno}
*Eu como usuário administrador quero cancelar o plano de contrato de um aluno,
 ou seja, fazer a recisão do mesmo.

\subsection[Alterar Plano]{Alterar Plano}
*Eu como usuário administrador quero mudar o plano de um aluno, seja na
quantidade de vezes por semana ou no período do plano.

\subsection[Cadastrar Aluno]{Cadastrar Aluno}
Eu como usuário administrador quero cadastrar novos alunos no sistema.

\begin{quote}
Critérios de Aceitação:
    \begin{itemize}
        \item O sistema deve permitir o cadastro de clientes com os dados: nome, profissão,
        data de nascimento, endereço, CEP, telefone residencial, celular, e-mail, CPF,
        RG e status.
    \end{itemize}
\end{quote}

\subsection[Alterar Dados de Aluno]{Alterar Dados de Aluno}
Eu como usuário administrador quero alterar os dados cadastrais dos alunos a fim
de manter um registro atualizado deles.

\begin{quote}
Critérios de Aceitação:
    \begin{itemize}
        \item O sistema deve permitir a edição de um ou mais dados de clientes como: nome,
        profissão, data de nascimento, endereço, CEP, telefone residencial, celular,
        e-mail, CPF e RG.
    \end{itemize}
\end{quote}

\subsection[Alterar Status de Aluno]{Alterar Status de Aluno}
Eu como usuário administrador quero ativar ou inativar um aluno no sistema
quando chegar ao fim de seu plano de contrato, de acordo com a renovação ou não.

\begin{quote}
Critérios de Aceitação:
    \begin{itemize}
        \item O sistema deve permitir a edição do \textsl{status} (ativo/não ativo) de um aluno
        previamente selecionado.
    \end{itemize}
\end{quote}

\subsection[Visualizar Perfil de Aluno]{Visualizar Perfil de Aluno}
*Eu como usuário administrador quero visualizar o perfil do aluno que contém suas
informações pessoais e plano de contrato.

\subsection[Gerar Relatório com os Aniversariantes do Mês]{Gerar Relatório com os Aniversariantes do Mês}
*Eu como usuário administrador quero visualizar a data de aniversário de todos os
alunos aniversariantes do mês para que possa desejá-los feliz aniversário.

\subsection[Buscar Alunos no Sistema]{Buscar Alunos no Sistema}
Eu como usuário administrador quero pesquisar alunos por nome.

\begin{quote}
Critérios de Aceitação:
    \begin{itemize}
        \item O sistema deve mostrar a lista de clientes que contenha um nome ou pedaço de
        nome, digitado em um campo existente;
        \item Dentro da busca o sistema deve permitir clicar para editar os clientes que
        apareceram.
    \end{itemize}
\end{quote}
