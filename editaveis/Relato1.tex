\section[Relato de experiência de execução do trabalho]{Relato de experiência de execução do trabalho}
Este trabalho se refere a modelagem de processo de um software para a empresa
Pilates Fit Club, ele foi divido em duas partes ao longo do semestre. A primeira
parte se refere ao planejamento do modelo de processo e a segunda parte se
refere a execução deste.

Toda pesquisa teórica é feita no relatório 1 (primeira parte). A escolha da
abordagem é feita nessa fase, analisando o que é melhor para a empresa, cliente
e grupo. Como por exemplo o fato de nossa cliente possuir pouco tempo para
reuniões nos levou a escolha da metodologia ágil, que proporciona reuniões
curtas e frequentes. Nessa fase também é feita a modelagem de processo e o
detalhamento de processos, sub-processos, atividades e eventos.

Também no relatório 1 é feita uma comparação entre os modelos de maturidade CMMI
e MPS-Br, para que seja selecionado um para implementar no projeto. As técnicas
de elicitação a serem utilizadas pelo grupo durante a segunda fase do projeto
são levantadas e explicitadas ainda nessa primeira fase.

Por último, mas não menos importante, um comparativo de ferramentas de gestão de
requisitos é feita para saber qual a que se encaixa melhor de acordo com o que
o grupo deseja executar.

Na segunda parte(relatório 2), temos a execução de tudo que foi planejado na
fase anterior. São levantados épicos, \textsl{features}, \textsl{user stories},
são definidas também \textsl{sprints} e \textsl{releases}. Por último temos a
implementação(código) do sistema planejado.

Ao utilizar o SAFe, \textsl{framework} baseado na metodologia ágil,
dividimos o projeto em nível de portifólio, programa e time. É feita uma reunião
com a cliente para levantamento das necessidades do estabelecimento,
logo em seguida o grupo levanta épicos, \textsl{features} e
\textsl{user stories}. Após o levantamento é feita outra runião com a cliente
para checar se é isso o mesmo o esperado por ela e se time e empresa estão
"falando a mesma liguagem". A priorização é feita na etapa seguinte após o grupo
se certificar de que tudo que foi levantado está correto.

Ao final da segunda fase temos a implementação do sistema de maneira parcial, de
acordo com o que foi priorizado, devido à falta de tempo para desenvolvimento
completo do código.
